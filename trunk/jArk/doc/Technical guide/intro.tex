\chapter*{Introduction}
\addcontentsline{toc}{chapter}{Introduction}
\label{cha:intro}
\section*{Group Composition}
\addcontentsline{toc}{section}{Group Composition}
\label{sec:group}
\begin{description}
  \item[Stefano Pongelli] [ stefano.pongelli@lu.unisi.ch ]
  \item[Thomas Selber] [ thomas.selber@lu.unisi.ch ]
\end{description}
\section*{The Game}
\addcontentsline{toc}{section}{The Game}
\label{sec:game}
Much like the game 'Pong', the player controls the "Vaus", a space vessel that acts as the game's "paddle" which prevents a ball from falling from the playing field, attempting to bounce it against a number of bricks. The ball striking a brick causes the brick to disappear. When all the bricks are gone, the player goes to the next level, where another pattern of bricks appear. There are a number of variations (bricks that have to be hit multiple times, flying enemy ships, etc.) and power-up capsules to enhance the Vaus (expand the Vaus, multiply the number of balls, equip a laser cannon, break directly to the next level, etc), but the gameplay remains the same.

\section*{Motivation}
\addcontentsline{toc}{section}{Motivation}
\label{sec:moti}

We chose to implement Arkanoid...