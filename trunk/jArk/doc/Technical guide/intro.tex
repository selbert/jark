\chapter*{Introduction}
\addcontentsline{toc}{chapter}{Introduction}
\label{cha:intro}
\section*{Group Composition}
\addcontentsline{toc}{section}{Group Composition}
\label{sec:group}
\begin{description}
  \item[Stefano Pongelli] [ stefano.pongelli@lu.unisi.ch ]
  \item[Thomas Selber] [ thomas.selber@lu.unisi.ch ]
\end{description}
\section*{The Game}
\addcontentsline{toc}{section}{The Game}
\label{sec:game}
Arkanoid is similar to 'Pong', the player controls the "Vaus", a space vessel that acts as the game's "paddle" which prevents a ball from falling from the playing field, attempting to bounce it against a number of bricks.\\
The ball hitting a brick causes the brick to disappear. When all the bricks are gone, the player goes to the next level, where another pattern of bricks appear.\\
Each brick could contain a Bonus which gives the player some particular ability, like an Ultra ball, a Rifle on the Vaus, .. 

\section*{Motivation}
\addcontentsline{toc}{section}{Motivation}
\label{sec:moti}

We choose to implement Arkanoid because it seemed to offer us a lot of freedom choosing which bonus implement and how, create a level editor, custom levels with strange shapes and so on. \\
Furthermore we could easily apply the Object Oriented feature of Java, for instance a Bonus is an abstract class with many subclasses, each of which is a specific Bonus (ie. FastBallBonus).\\
The same goes for the Vaus, for the Ball and for the Bricks.\\