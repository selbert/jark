\chapter*{Milestones}
\label{cha:milestones}
\addcontentsline{toc}{chapter}{Milestones}

\section*{M1, initial idea}
\label{sec:m1}
\addcontentsline{toc}{section}{M1, initial idea}
/* MODEL */\\
\\
// class to describe ball\\
class Ball\\
// class to describe the vaus/paddle\\
class Vaus\\
// class to describe a brick\\
class Brick\\
\\
birck types :\\
 - 0 : normal\\
 - 1 : double\\
 - 2 : triple\\
 - 3 : fixed\\
\\
// class that describes the position of the bricks on a level (maybe parsing files)\\
class Level\\
// class that contain information about the player\\
class Player\\
// class that holds information about bonus\\
class Bonus:\\
 - 0 : sticky ball(1)\\
 - 1 : slow ball(1)\\
 - 2 : long vaus(1)\\
 - 3 : explosion ball(2)\\
 - 4 : doubling ball(2)\\
 - 5 : laser(2)\\
 - 6 : ultra ball(3)\\
 - 7 : double laser(3)\\
 - 8 : missile (4)\\
 - 9 : the black box (5)\\
 - 10 : fast ball(1)\\
 - 11 : short vaus (1)\\
 - 12 : ghost ball (2)\\
 - 13 : block freeze (inefficient ball) (3)\\
 - 14 : reset vaus (3)\\
 - 15 : false balls (become true after time) (4)\\
 - 16 : death (5)\\
\\
// enum to hold the different types of bonus\\
enum BonusEnum\\
// class that holds information about the current state of the game\\
class Game\\
\\
/* CONTROLLER */\\
\\
// class to parse mouse/keyboard inputs\\
class InputParser\\
// enum with messages (maybe taken from a file?)\\
enum MessageEnum\\
// class to receive input and print outputs\\
class Console\\
\\
/* VIEW */\\
\\
// class to design the output (draw all)\\
class Field\\

\section*{M2: Ball implementation}
\label{sec:m2}
\addcontentsline{toc}{section}{M2: Ball implementation}
- Added inheritance (bonus, brick, vaus)\\
- Changed packet structure (issue 2 and 3)\\
- Removed state from vaus (issue 1)\\
- Solved issue 1 (google code).\\
\\
- issue: bouncing method.\\
	- solved with current and next position of the ball on the matrix.\\
\\
- issue: bonus returning w/o passing from ball (to lower the coupling)\\
	- bonus treated by game, ball returns the brick to explode and the game class lets the level class handle it and return a bonus.\\
\\
- project has to be run from bash in order to view the simulation (only the field is shown, the ball is invisible at the moment)\\
to run it: just press enter continuously, or type something to quit. You will see the bricks hit by the ball disappearing. \\
\\
*****\\
Pay attention: also if we have few tests, we are quit sure the implemented things are working properly.\\
Moreover this is an early stage of our project, we have to improve a lot of things and change a lot more.\\
The most important element at this time, is the ball. And this is tested.\\
*****\\

\section*{M3: Basic GUI}
\label{sec:m3}
\addcontentsline{toc}{section}{M3: Basic GUI}
For this Milestone we implemented a basic GUI with mouse and key event listener,and a main Thread which provides us with a sort of "tick" where we reprint the game and update everything.\\

Issues: how to compute the intersection between the ball (which is not a pixel anymore) and the bricks, \\
		how to make the ball bounce on the vaus\\
		how to structure the view (divided in panels, ...)\\

\section*{M4: Bonus and brick}
\label{sec:m4}
\addcontentsline{toc}{section}{M4: Bonus and brick}

Issues: .jar not working\\
\\
TODO:	we have to finish implementing some bonus\\
		add some listener\\
		add player support\\
		add ending event\\
		finish the gui\\
		add a level editor\\
		add an algorithm to create levels\\
		...\\
\\
Solved Issues:\\
- The view has to know the Model, and not the inverse. \\
  We had to refactor the project. At this moment Model and View are completely divided.\\
- The different images of each element (bonus, vaus, ball and brick) were inside the element, and each time they were needed they were created (new). Now the images are all in a class in View and there is only one instance of each image for all the elements (they are created wih the constructor of that class and only the reference is passed on).\\
- We had to add Listeners in order to apply the changes given by bonuses (ie. to the vaus)\\
\\
Implemented:\\
Bonuses: everything except for shooting vauses and sticky ball (lives are half implemented)\\

Brick types: \\
we have 4 different bricks: \\
default, destroyed the fist time it is hit;\\
persistent, which doesn't get destroyed;\\
resistent, needs to be hit 2 times\\
very resistent, three times.\\
\\
Images: each bonus/vaus/ball/brick has its image. \\
		Those images are created inside an hashmap in view.ImagesReference, and only a reference is given to the GamePanel which draws them.\\
\\
Swing Timer instead of Threads\\

Furthermore, we added more tests, fixed a lot of bugs, and refactored the entire project.\\

\section*{M5: Level editor}
\label{sec:m5}
\addcontentsline{toc}{section}{M5: Level editor}
Done:\\
- lots of bugfixes.\\
- Level Editor (with save/load/test capabilities).\\
- refactoring (listeners).\\
- GUI (general).\\
- implemented all bonuses.\\
- implemented bonus life span in the game frame.\\
- implemented general info (lives, points, ..).\\
- game over/next level (random at the moment)/replace ball after loosing life.\\
\\
Problems encountered (solved): \\
- find a "good" way to implement the editor.\\
- display taken bonuses and their duration on main screen.\\
\\
Problems: \\
- code duplication (ie. bonus strings).\\

\section*{M6: }
\label{sec:m6}
\addcontentsline{toc}{section}{M6: Ball implementation}